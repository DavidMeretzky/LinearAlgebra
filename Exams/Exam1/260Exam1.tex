\documentclass{article}

\usepackage{algorithmic, amsmath, amsthm, amsfonts, amssymb,commath, enumerate, tikz, tikz-cd, color, mathrsfs} %tikz is for drawing lattices %tikz-cd is for commutative diagrams
															%color is for making notes in red 
															%mathrsfs is for power set font
%\usepackage[mathscr]{eucal} %mathscr gives nice script fonts

\newtheoremstyle{problemstyle}  % <name> This is my problemstyle. use begin{problem}.
        {12pt}                                               % <space above>
        {}                                               % <space below>
        {}%\itshape}                               % <body font>
        {}                                                  % <indent amount}
        {\bfseries}                 % <theorem head font>
        {\normalfont\bfseries.}         % <punctuation after theorem head>
        {.5em}                                          % <space after theorem head>
        {}                                                  % <theorem head spec (can be left empty, meaning `normal')>


\theoremstyle{problemstyle}

\newtheorem{problem}{Problem}

\theoremstyle{problemstyle}

\newtheorem{solution}{Solution}

\theoremstyle{problemstyle}

\newtheorem{definition}{Definition}


\title{ \vspace{-10ex} %uncomment to remove vertical space
%title of assignment goes here e.g. "Math 721 Homework 3"
Math 260 Exam 1
}


\author{David L. Meretzky
}


\date{%date assignment is due goes here
Monday October 1st, 2018
} 


\renewcommand*{\thefootnote}{$\dagger$} %changes default footnote marking to a dagger instead of a number (numbers are sometimes mistaken for citations)

\begin{document}

\maketitle

In problems $1$ and $2$ your wording may be different than that of book. I am looking for correctness in the concepts. Problems $1-3$ are worth $70$ points.  The test is out of $100$. 

\begin{problem}
Give definitions for the following terms:
\begin{enumerate}
\item vector space 
\item subspace 
\item give the $3$ conditions for a subspace
\item sum of subspaces
\item direct sum of subspaces
\item linear combination of a list of vectors
\item span of a list of vectors
\item linear independence of a list of vectors
\item basis for a vector space 
\item dimension of a vector space
\end{enumerate}
\end{problem}

\begin{problem}
State and prove the Linear Dependence Lemma.
\end{problem}

\begin{problem}
Suppose $p_0, p_1,..., p_m$ are polynomials in $\mathcal{P}_m$$($\textbf{F}$)$ such that $p_j(2) = 0$ for each $j$. Prove that $p_0, p_1,..., p_m$ is not linearly independent in $\mathcal{P}_m$$($$\textbf{F}$$)$.
\end{problem}

Do enough problems to reach the remaining $30$ points:

\begin{problem}
(10 points) Let $$U = \{(x,y,x+2y, 4x-y, 3y) \in \textbf{F}^5 \ : \ x,y \in \textbf{F}\}$$ Find a basis for $U$. Extend this basis to a basis of $\textbf{F}^5$ and then use this to find a subspace $W$ such that $W \oplus U = \textbf{F}^5$.  
\end{problem}

\begin{problem}
(5 points) What is the dimension of $\mathcal{P}_m$?
\end{problem}

\begin{problem}
(5 points) Prove $(1,2),(3,5)$ is a basis of $\textbf{F}^2$. 
\end{problem}

\begin{problem}
(10 points) Prove $(1,-1,0),(1,0,-1)$ is a basis of $\{(x,y,z) \in \textbf{F}^3 \ : \ x+y+z =0 \}$
\end{problem}

\begin{problem}
(10 points) Suppose $U$ and $W$ are subspaces of $V$. Then $U + W$ is a direct sum if and only if $U\cap W = \{0\}$.
\end{problem}

\begin{problem}
(5 points) Suppose $U_1$ and $U_2$ are subspaces of $V$. Prove that the intersection $U_1 \cap U_2$ is a subspace of $V$. 
\end{problem}

\begin{problem}
(5 points) Verify both distributive properties in the definition of a vectorspace for $\textbf{F}^3$.
\end{problem}



\end{document}
