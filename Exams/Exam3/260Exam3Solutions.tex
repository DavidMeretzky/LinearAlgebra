\documentclass{article}

\usepackage{algorithmic, amsmath, amsthm, amsfonts, amssymb,commath, enumerate, tikz, tikz-cd, color, mathrsfs} %tikz is for drawing lattices %tikz-cd is for commutative diagrams
															%color is for making notes in red 
															%mathrsfs is for power set font
%\usepackage[mathscr]{eucal} %mathscr gives nice script fonts

\newtheoremstyle{problemstyle}  % <name> This is my problemstyle. use begin{problem}.
        {12pt}                                               % <space above>
        {}                                               % <space below>
        {}%\itshape}                               % <body font>
        {}                                                  % <indent amount}
        {\bfseries}                 % <theorem head font>
        {\normalfont\bfseries.}         % <punctuation after theorem head>
        {.5em}                                          % <space after theorem head>
        {}                                                  % <theorem head spec (can be left empty, meaning `normal')>


\theoremstyle{problemstyle}

\newtheorem{problem}{Problem}

\theoremstyle{problemstyle}

\newtheorem{solution}{Solution}

\theoremstyle{problemstyle}

\newtheorem{example}{Example}

\theoremstyle{problemstyle}

\newtheorem{definition}{Definition}


\title{ \vspace{-10ex} %uncomment to remove vertical space
%title of assignment goes here e.g. "Math 721 Homework 3"
Math 260 Exam 3 Take Home Solutions
}


\author{David L. Meretzky
}


\date{%date assignment is due goes here
Friday Decmeber 7th, 2018
} 


\renewcommand*{\thefootnote}{$\dagger$} %changes default footnote marking to a dagger instead of a number (numbers are sometimes mistaken for citations)

\begin{document}

\maketitle

\noindent
Problems 1-5 are worth 18 points each. Problem 6 is worth 10 points. The test is out of 100 points. 

\begin{definition}
Let $V$ be a real or complex vectorspace, $\textbf{F} = \textbf{R}$ or $\textbf{C}$. A norm on $V$ is a real-valued function $||\ \  ||:V \rightarrow \textbf{R}$ such that 
\begin{enumerate}
\item for any non-zero vector $v \in V$, $||v|| > 0$, 
\item for any scalar $\alpha \in \textbf{F}$, $||\alpha v|| = |\alpha|||v||$ for all $v \in V$,
\item for any $u,v \in V$ $||u+v|| \leq ||u|| + ||v||$
\end{enumerate}
We call $V$ a normed linear space. 
\end{definition}


\flushleft
Let $B = e_1,...,e_n$ be the usual basis for $\textbf{F}^n$. For instance, we know that $\textbf{R}^n$ has the usual euclidean norm: for $v \in  \textbf{F}^n$, $v = a_1e_1 +...+a_ne_n$, define 
\begin{equation}||v|| = (\sum_{i = 1}^na_i^2)^{\frac{1}{2}}\end{equation}

\begin{example}
$$||(1,-1)|| = \sqrt{1^2+(-1)^2}$$
\end{example}

Clearly $\textbf{F}^n$ is a normed linear space. You will show that if $V$ is finite dimensional then $\mathscr{L}(V)$ is a normed linear space. 

\begin{definition}
Let $V$ be a finite dimensional normed linear space and let  $T \in \mathscr{L}(V)$. Define the operator norm of $T$ to be the smallest number $M$ such that $||Tv|| \leq M||v||$ for any $v \in V$. We will write $||T||$ to mean that smallest number $M$, the operator norm. 
\end{definition}

\flushleft
Notice that the norms in the expression $||Tv|| \leq M||v||$ are the norm that $V$ was born with. That is, this definition only makes sense if $V$ has a norm. 

\begin{problem}
Let $B = e_1,...,e_n$ be an orthonormal basis for $V$ a normed linear space of dimension $n$. Let $T \in \mathscr{L}(V)$. Let $m = Max\{||Te_1||,||Te_2||,...,||Te_n||\}$. That is, $m$ is the length of the longest vector in the list $Te_1,...,Te_n$. Prove that for any unit-length vector $v \in V$, $||Tv|| \leq mn$. 
\end{problem}

\begin{solution}
We have that $v = \sum_{i=1}^na_ie_i$ moreover, the requirement that $||v|| = 1$ means that $\sum_{i=1}^na_i^2 = 1$, and thus for each $i$, $|a_i| \leq 1$.   We compute, 
\begin{equation}
||Tv|| = ||T(\sum_{i = 1}^na_ie_i)|| = ||\sum_{i = 1}^na_iTe_i|| \leq \sum_{i = 1}^n|a_i|||Te_i|| \leq \sum_{i = 1}^n||Te_i|| \leq \sum_{i = 1}^nm = nm.
\end{equation}
\end{solution}

\begin{problem}
Let $B = e_1,...,e_n$ be an orthonormal basis for $V$ a normed linear space of dimension $n$. Let $T \in \mathscr{L}(V)$. Show that the operator norm of $T$ exists and is finite. (I am asking you to show that taking any $v \in V$, show that there exists a number $K$ such that $||Tv|| \leq $K$||v||$.) Hint: Use the conclusion of the previous problem. Hint: maybe the problem is easier if you assume $||v|| = 1$? 
\end{problem}

\begin{solution}
Assume that $||v|| = 1$, then $||Tv|| \leq nm = nm||v||$. Letting $K=nm$ we are finished. Now suppose that we have any $u \in V$,  
\begin{equation*}
||Tu|| = ||u||||(\frac{1}{||u||}Tu)|| = ||u||||T(\frac{u}{||u||})||
\end{equation*}
A vector divided by it's norm is of norm $1$. Let $v = \frac{u}{||u||}$. Then $||v|| = 1$. Furthermore, 
\begin{equation*}
||Tu|| = ||u||||Tv|| \leq ||u||nm||v|| = K||u|| 
\end{equation*}
where $K = nm||v|| = nm$. 
\end{solution}

Now that we know the operator norm exists and is finite:

\begin{problem}
Show that the operator norm is a norm (satisfies definition 1) on $\mathscr{L}(V)$ for a finite dimensional normed linear space $V$.  
\end{problem}

\begin{solution}
We must show that for $T \in \mathscr{L}(V)$ such that $T \neq 0 \in \mathscr{L}(V)$, $||T|| > 0$. 
Since $T \neq 0$, there exists at least one vector $v \in V$ such that $Tv \neq 0 \in V$. Thus 
\begin{equation*}
0<||Tv|| \leq ||T||||v||.
\end{equation*}
Since $Tv \neq 0$ it follows that $||v|| > 0$ and since $0< ||T|| ||v||$ we must have that $||T||>0$ as desired.\\
\vspace{3mm}
Next we must show that for $T \in \mathscr{L}(V)$, $||\alpha T|| = |\alpha| ||T||$

Given any $v \in V$ by the definition of the operator norm for the operator $\alpha T$ we have   $$||\alpha T v|| \leq ||\alpha T||||v||.$$ However, $$||\alpha T v||  = |\alpha|||Tv|| \leq |\alpha|||T|| ||v||.$$ Thus, $||\alpha T|| \leq |\alpha|||T||$. \\
\vspace{3mm}
To show the reverse inequality, note that for $\alpha \neq 0$: $$||Tv|| = ||\alpha T \frac{1}{\alpha} v|| \leq ||\alpha T||||\frac{1}{\alpha}v|| = |\frac{1}{\alpha}|||\alpha T||||v||$$ from which it follows that $$||T|| \leq |\frac{1}{\alpha}|||\alpha T||$$ and thus $$|\alpha|||T|| \leq ||\alpha T||.$$ The desired equality $$|\alpha|||T|| = ||\alpha T||$$ is obtained. The case $\alpha = 0$ is immediate.\\
\vspace{3mm}
For any $T,S \in \mathscr{L}(V)$ and any $v \in V$, we compute $$||(S+T)(v)|| = ||Sv + Tv|| \leq ||Sv|| + ||Tv|| \leq ||S||||v|| + ||T||||v|| = (||S|| + ||T||)||v||$$ from which it follows that $$||S+T|| \leq ||S|| + ||T||$$ where the second inequality above comes from the triangle inequality for the norm on $V$.
\end{solution}

%What does the operator norm have to do with the largest eigenvalue?\\
Let $T$ be an invertable linear operator of a finite dimensional normed linear space $V$. 

\begin{problem}
Let $v$ be an eigenvector for $T$ with eigenvalue $\lambda$. Prove that $||Tv|| = |\lambda|||v||$. Prove that $|\lambda| \leq ||T||$. 
\end{problem}

\begin{solution}
We have $||Tv|| = ||\lambda v|| = |\lambda|||v||.$ Now compute $$|\lambda|||v|| = ||Tv|| \leq ||T||||v||$$ from which it follows that $|\lambda| \leq ||T||$. 
\end{solution}

\begin{problem}
Suppose $T \in \mathscr{L}(V)$ is invertible. Suppose $\lambda \in \textbf{F}$ with $\lambda\neq 0$. Prove that $\lambda$ is an eigenvalue of $T$ if and only if $\frac{1}{\lambda}$ is an eigenvalue of $T^{-1}$. 
\end{problem}

\begin{solution}
Let $v$ be an eigenvector for $\lambda$. Then $T^{-1}Tv = T^{-1}\lambda v = \lambda T^{-1}v$. But since $T^{-1}Tv = v$, we must have that $\lambda T^{-1}v = v$ from which it follows that $T^{-1}v = \frac{1}{\lambda}v$. Since all steps of the proof hold with equality, the result holds in both directions. 
\end{solution}

\begin{problem}
Let $V$ be a finite dimensional inner product space over $\textbf{R}$. What can you say about the relationship between the induced norm on $V$ and the operator norm on $\mathscr{L}(V,\textbf{R})$? Define the operator norm on $\mathscr{L}(V,\textbf{R})$ by letting $||\phi||$ for $\phi \in \mathscr{L}(V,\textbf{R})$ be the smallest number $M$ such that $|\phi(v)| \leq M||v||$ for all $v\in V$.\\ 

Hint: The Reisz Representation Theorem gives a nice association: for every $\phi \in \mathscr{L}(V,\textbf{R})$ there exists a unique $v \in V$ such that $\langle \ \  ,v\rangle$ is equal to $\phi( \ \ )$.  
\end{problem}

\begin{solution}
Pick any $\phi$. There exists a unique $v \in V$ such that $\langle \ \  ,v\rangle$ is equal to $\phi( \ \ )$. 
Now for any $u \in V$, we compute $$|\phi(u)| = |\langle u,v\rangle| \leq ||u||||v||$$ by the The Reisz Representation Theorem and the Cauchy-Schwarz Inequality. It must follow that $$||\phi|| \leq ||v||.$$ To show the reverse inequality note that $$|\phi(v)| = |\langle v,v\rangle| = ||v||^2 \leq ||\phi||||v||$$ from which it follows that $$||v|| \leq ||\phi||$$ and furthermore that $$||\phi|| = ||v||.$$
\end{solution}

\end{document}
