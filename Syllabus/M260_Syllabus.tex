\documentclass{article}
\usepackage{verbatim}
\begin{document}
	{\bfseries Hunter College
		\newline Fall 2018
		\newline Math 26000: Linear Algebra $\;\;\;\;$ 4 hrs, 4 cr. 
		\newline Mo We 7:35pm-9:25pm, Hunter West 411
		\newline Instructor: David Meretzky
		\newline Email: dm594@hunter.cuny.edu
		\newline Office Hours: See Below
		}
	
	REQUIRED TEXT: \textit{Linear Algebra Done Right}, Third Edition, by Sheldon Axler ISBN: 978-3-319-11079-0
	
	Course Outline: We will cover at least chapters 1-6,8,10 of the text. With any additional time we may cover some topics from other chapters. 
	
	Prerequisites: This course takes an abstract approach to Linear Algebra, so the course is proofs-based with very little computation. Being comfortable with proofs is essential to passing the course. A prerequisite for this course is the proof workshop Math 156, so we assume that everyone is familiar with proofs at a basic level. 
	
	Some things from the proofs course that should be familiar are proof structures  (\textit{if, then} statements, \textit{if and only if} statements, negations, converses and contrapositives, proofs by contradiction, existence and uniqueness proofs, and proofs by strong/weak induction) and sets/functions  (definition of a set, how do we notate them, how do we show that one set is a subset of another, how do we show two sets are equal, what is a function between two sets, what is the composition of two functions). There are many references for this material, one of which is \textit{How to Prove It: A Structured Approach, 2nd Edition} by Daniel J. Velleman.
	\begin{comment}\begin{enumerate} \item proof structures \begin{itemize}\item if, then \item if and only if \item converse, contrapositive \item contradiction \item existence and uniqueness \item induction\end{itemize} \item \begin{itemize} \item what are sets, and how do we notate them? \item how to show that one set is a subset of another set \item how to show two sets are equal \item what is a function between two abstract sets? \item composition of functions? \end{itemize}\end{enumerate}\end{comment}
	
	Homework: I will assign exercises at the end of lecture. Homework is due at the begining of class. I will post solutions after the homework is due. Since I will post solutions, I will not always return homework.  Make sure you have a copy. You should always check your work against the solutions. Homework grading will be binary, i.e, Done or Not Done. 
	
	Exams: There will be two midterms and a final exam. The final will be cumulative. The final time will be announced later in the course. On an exam, one might be asked to state definitions, state and prove theorems, and solve problems similar to the homework.
	
	Attendance: Attendance is not mandatory, but this course is very difficult so you will find yourself in a rough spot if you miss too many classes. If you do miss a class due to an emergency, you should get the notes from a classmate. Attendance will be taken for record keeping. If you have notified me that you will be absent you may email me a scanned pdf of your homework by the begining of class. 
	
	How to Study: As mentioned previously, this material is very abstract. In general, there are no recipes for proofs, so emphasis is placed not on following a pattern but rather on understanding the material.  The first step in understanding the material is understanding the definitions. Being able to derive simple facts from these definitions demonstrates familiarity with the objects being defined. One can obtain this fluency only by spending a lot of time with a pen and paper, working out the details of the text. Mathematics can be learned only by doing, not simply by reading the textbook. If you breeze through a page in less than an hour, you are moving too fast. In addition to working out the details of the text, you should try to make examples for yourself and solve lots of exercises.
	
	%Presentations: I am giving the opportunity for students to give a presentation (for extra credit) on material related to linear algebra. The topic should be chosen by the student and a short paragraph-long proposal should be formed and presented to me for approval. There are many ``starred'' sections in the text which we will not cover but are interesting topics for presentation. There are many applications of linear algebra to the sciences which also serve as interesting topics for presentation. There are no formal guidelines for the presentation but keeping in contact with me will ensure that the project goes smoothly.
	
	Grades: Homework 20\% 
	\newline\indent$\;\;\;\;\;\;\;\;\;\;\;\;$ Midterms 20\% each 
	\newline\indent$\;\;\;\;\;\;\;\;\;\;\;\;$ Final Exam 40\%
	
	Office Hours: I am not assigned office hours. However, I will try to be available before and after each class for any questions.
\end{document}