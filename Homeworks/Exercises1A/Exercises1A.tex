\documentclass{article}

\usepackage{algorithmic, amsmath, amsthm, amsfonts, amssymb,commath, enumerate, tikz, tikz-cd, color, mathrsfs} %tikz is for drawing lattices %tikz-cd is for commutative diagrams
															%color is for making notes in red 
															%mathrsfs is for power set font
%\usepackage[mathscr]{eucal} %mathscr gives nice script fonts

\newtheoremstyle{problemstyle}  % <name> This is my problemstyle. use begin{problem}.
        {12pt}                                               % <space above>
        {}                                               % <space below>
        {\itshape}                               % <body font>
        {}                                                  % <indent amount}
        {\bfseries}                 % <theorem head font>
        {\normalfont\bfseries.}         % <punctuation after theorem head>
        {.5em}                                          % <space after theorem head>
        {}                                                  % <theorem head spec (can be left empty, meaning `normal')>


\theoremstyle{problemstyle}

\newtheorem{problem}{Problem}

\theoremstyle{problemstyle}

\newtheorem{solution}{Solution}


\title{ \vspace{-10ex} %uncomment to remove vertical space
%title of assignment goes here e.g. "Math 721 Homework 3"
Math 260 Exercises 1.A
}


\author{David L. Meretzky
}


\date{%date assignment is due goes here
Tuesday August 21th, 2018
} 


\renewcommand*{\thefootnote}{$\dagger$} %changes default footnote marking to a dagger instead of a number (numbers are sometimes mistaken for citations)

\begin{document}

\maketitle

\begin{problem}
Suppose $a$ and $b$ are real numbers, not both $0$. Find real numbers $c$ and $d$ such that $\frac{1}{(a+bi)} = c+di$.
\end{problem}

This problem shows that every complex number has a multiplicative inverse. We can then be sure that our definition of division for complex numbers from class will hold, that is for $\alpha \in \mathbb{C}$ where $\alpha \neq 0$. This is exactly the requirement that for $\alpha = a+bi$, $a$ and $b$ are real numbers which are not both $0$.

\begin{problem}
Verify properties of $\mathbb{C}$: 

Note from class: Think of $\lambda$ as a scalar or number like something in $\mathbb{F}$ and think of $\alpha$, $\beta$, and $\gamma$ as vectors in $\mathbb{F}^n$, (in this case the vectors are actually in $\mathbb{C}^n$) for $n = 1$.  

Show that for all $\alpha$, $\beta$, $\gamma$ $\lambda$ in $\mathbb{C}$. 
\begin{enumerate}
\item $\alpha + \beta = \beta + \alpha$  commutativity of addition
\item $(\alpha+\beta)+\gamma = \alpha+(\beta + \gamma)$ associativity of addition
\item $(\alpha\beta)\gamma = \alpha(\beta\gamma)$ commutativity of multiplication
\item Show that for every $\alpha$ there is a $\beta$ so that $\alpha + \beta = 0$
\item Show that for every $\alpha$ there is a $\beta$ so that $\alpha\beta = 1$
\item $\lambda(\alpha+\beta) = \lambda\alpha +\lambda\beta$
\end{enumerate}
\end{problem}


\begin{problem}
Find $x \in \mathbb{R}^4$ such that $(4,-3,1,7)+2x = (5,9,-6,8)$.
\end{problem}


\begin{problem}
Explain why there does not exist $\lambda \in \mathbb{C}$ such that $\lambda(2-3i,5+4i,-6+7i) = (12-5i,7+22i,-32-9i)$.
\end{problem}


\begin{problem}
Verify properties of $\mathbb{F}^n$: 
Let $x,y,z\in\mathbb{F}^n$ be vectors and $a,b\in\mathbb{F}$ be scalars
\begin{enumerate}
\item show associativity of addition $(x+y)+z = x+(y+z)$
\item show associativity of scalar multiplication $(ab)x = a(bx)$
\item show there is a multiplicative identity $1x=x$ 
\item show distributivity of scalar multiplication (vector side) $a(x+y) = ax + ay$
\item show distributivity of scalar multiplication (scalar side)  $(a+b)x = ax + bx$
\end{enumerate}
\end{problem}


\end{document}
