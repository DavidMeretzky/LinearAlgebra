\documentclass{article}

\usepackage{algorithmic, amsmath, amsthm, amsfonts, amssymb,commath, enumerate, tikz, tikz-cd, color, mathrsfs} %tikz is for drawing lattices %tikz-cd is for commutative diagrams
															%color is for making notes in red 
															%mathrsfs is for power set font
%\usepackage[mathscr]{eucal} %mathscr gives nice script fonts

\newtheoremstyle{problemstyle}  % <name> This is my problemstyle. use begin{problem}.
        {12pt}                                               % <space above>
        {}                                               % <space below>
        {\itshape}                               % <body font>
        {}                                                  % <indent amount}
        {\bfseries}                 % <theorem head font>
        {\normalfont\bfseries.}         % <punctuation after theorem head>
        {.5em}                                          % <space after theorem head>
        {}                                                  % <theorem head spec (can be left empty, meaning `normal')>


\theoremstyle{problemstyle}

\newtheorem{problem}{Problem}

\theoremstyle{problemstyle}

\newtheorem{solution}{Solution}


\title{ \vspace{-10ex} %uncomment to remove vertical space
%title of assignment goes here e.g. "Math 721 Homework 3"
Math 260 Exercises 1.B 
}


\author{David L. Meretzky
}


\date{%date assignment is due goes here
Tuesday August 21th, 2018
} 


\renewcommand*{\thefootnote}{$\dagger$} %changes default footnote marking to a dagger instead of a number (numbers are sometimes mistaken for citations)

\begin{document}

\maketitle

\begin{problem}
Prove that $-(-v) = v$ for every $v \in V$ 
\end{problem}

\begin{problem}
Suppose $a \in \mathbb{F}$, $v \in V$ and $av = 0$. Prove that $a = 0$ or $v = 0$. 
\end{problem}

\begin{problem}
Suppose $v,w \in V$. Explain why there exists a unique $x \in V$ such that $v + 3x = w$.
\end{problem}

\begin{problem}
The empty set is not a vectorspace. The empty set fails to satisfy only one of the requirements listed in 1.19. Which one?
\end{problem}

\begin{problem}
Show that in the definition of a vectorspace (1.19), the additive inverse condition can be replaced with the condition that $0v = 0$ for all $v \in V$. Here $0$ on the left side is the number $0$, and the $0$ on the right side is the additive identity of $V$. 
\end{problem}

\begin{problem}
Let $\infty$ and $-\infty$ denote two distinct objects, neither of which is in $\mathbb{R}$. Define an addition and scalar multiplication on $\mathbb{R}\cup \{\infty\}\cup \{-\infty\}$ as you could guess from the notation. Specifically, the sum and product of two real numbers is as usual, and for $t \in \mathbb{R}$ define $t\infty
 = -\infty$ if $t < 0$, $t\infty = 0$ if $t = 0$ and $t\infty = \infty$ for $t>0$. Also $t+\infty = \infty +t =  \infty$, $t+(-\infty) = (-\infty) +t =-\infty$. $\infty+\infty =\infty$, $(-\infty)+(-\infty) =(-\infty)$, $0 = (-\infty)+\infty$.\\
Is $\mathbb{R}\cup \{\infty\}\cup \{-\infty\}$ a vectorspace over $\mathbb{R}$? Explain.
 \end{problem}

\end{document}
